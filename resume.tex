\documentclass[%singlesided,
               doublesided,
               paper=a4,
               fontsize=10pt
              ]{my-resume}


%%%%%%%%%%%%%%%%%%%%%%%%%%%%%%%%%%%%%%%%%%%%%%%%%%%%%%%%%%%%%%%%%%%%%%%%%%%%%%%%
% set geometry
%%%%%%%%%%%%%%%%%%%%%%%%%%%%%%%%%%%%%%%%%%%%%%%%%%%%%%%%%%%%%%%%%%%%%%%%%%%%%%%%

\setlength\highlightwidth{8cm}
\setlength\headerheight{4cm}            % note that margintop gets added to this value, i.e. the header bar is 5cm
\setlength\marginleft{1cm}
\setlength\marginright{\marginleft}      % needs to be 1.5 times to be actually equal. why?
\setlength\margintop{1cm}
\setlength\marginbottom{1cm}


%%%%%%%%%%%%%%%%%%%%%%%%%%%%%%%%%%%%%%%%%%%%%%%%%%%%%%%%%%%%%%%%%%%%%%%%%%%%%%%%
% FONTS
%%%%%%%%%%%%%%%%%%%%%%%%%%%%%%%%%%%%%%%%%%%%%%%%%%%%%%%%%%%%%%%%%%%%%%%%%%%%%%%%

\RequirePackage{fontspec}
\setmainfont{Carlito}


%%%%%%%%%%%%%%%%%%%%%%%%%%%%%%%%%%%%%%%%%%%%%%%%%%%%%%%%%%%%%%%%%%%%%%%%%%%%%%%%
% COLORS
%%%%%%%%%%%%%%%%%%%%%%%%%%%%%%%%%%%%%%%%%%%%%%%%%%%%%%%%%%%%%%%%%%%%%%%%%%%%%%%%

\colorlet{highlightbarcolor}{lightgray}
\colorlet{headerbarcolor}{darkgray}

\colorlet{headerfontcolor}{white}
\colorlet{accent}{awesome-red}
\colorlet{heading}{black}
\colorlet{emphasis}{black}
\colorlet{body}{black}


%%%%%%%%%%%%%%%%%%%%%%%%%%%%%%%%%%%%%%%%%%%%%%%%%%%%%%%%%%%%%%%%%%%%%%%%%%%%%%%%
% set document
%%%%%%%%%%%%%%%%%%%%%%%%%%%%%%%%%%%%%%%%%%%%%%%%%%%%%%%%%%%%%%%%%%%%%%%%%%%%%%%%


\begin{document}

\name{Agustín Damián Ulloa}
\tagline{Hola! Me llamo Agustín. Tengo 18 años y estoy estudiando técnicatura en computación.\\
Me interesa el software libre, y aprecio mucho trabajar con Linux.\\
Soy amigable y puedo adaptarme a cualquier entorno (tanto social como tecnológico).\\
Busco aprender constantemente y deseo contribuir a mi bienestar y al de quienes me rodean.}
\photo[round]{picture.jpg}{\dimexpr \headerheight-\marginbottom}   % make photo exactly match the header with margintop/marginright/marginbottom as margin

\makeheader

\highlightbar{

    \section{Habilidades}
    
    \vspace{0.5em}
    \skillsection{Sistemas Operativos}
    \skill{GNU/Linux}{4}
    \skill{macOS}{1}
    (nunca usé macOS, pero no tendría problema con ello.)\\
    \skill{Microsoft Windows}{4}
    
    \vspace{0.5em}
    \skillsection{Software y Herramientas}
    \skill{Docker}{1}
    \skill{neovim}{2}
    \skill{GIMP}{2}
    \skill{Office}{4}
    
    \section{Lenguajes}
    
    \vspace{0.5em}
    \skillsection{Humanos (escritos, hablados y leídos)}
    \skill{Español}{5}
    \skill{Inglés}{4}
    \bigskip

    \skillsection{Máquina}
    \skill{C}{1}
    \skill{Python}{1}
    \skill{Bash}{2}
    \skill{HTML/CSS}{2}
    \skill{JavaScript}{1}
    \skill{LaTeX}{2}
    \skill{Beamer}{1}

    \section{Contacto}
    
    \email{agus@agustinulloa.xyz}
    \phone{(+54) 11 5990-7299}
    \location{Ramallo 4445, 1430 CABA}
    \vspace{0.5em}
    \homepage{agustinulloa.xyz}{https://www.agustinulloa.xyz}
    \matrixorg{@invertwhite:matrix.org}
    \github{@gh0ste}{https://github.com/gh0ste}
    \linkedin{Agustín Ulloa}{https://www.linkedin.com/in/agustín-ulloa-a2569a21b}
    \orcid{0000-0002-3033-6184}{https://www.orcid.org/0000-0002-3033-6184}
    
    \section{Página Web}
    \includegraphics[width=0.45\textwidth]{qr.png}

}
\mainbar{
  
%    \section{About this template}
%    Section are set in bold face. An optional parameter of \texttt{\textbackslash section} takes a symbol to add in front of the text. This option is used in the jobs and education sections below.
    
    \section[\faGears]{Historial de trabajo}
    No tengo experiencia laboral de ningún tipo, pero sí tengo muchísimas ganas de incursionar en el mundo laboral. 

    Quiero tener contacto con verdaderos profesionales, aprender nuevas cosas constantemente, mejorar y vivir 
    la experiencia de trabajar en serio.
    
    \section[\faMortarBoard]{Educación}
    \job{03/2016 - 8/2021}
        {\href{https://www.et21.com.ar}{ET N°21 "Fragata Escuela Libertad", \\CABA}}
        {Técnico en Computación}
        \par{Aún cursando la secundaria, pero muy cerca de terminar el último año.}

    \job{¿2022 - ...?}
        {\href{http://www.fi.uba.ar/}{Universidad Buenos Aires (UBA), \\CABA}}
        {Ingeniería en Informática}
        \par{Tengo planeado comenzar con esta carrera una vez finalice la secundaria.}

    \section{Aptitudes}
    \smallskip
    \tag{adaptable}
    \tag{ordenado}
    \tag{creativo}
    \tag{honesto}
    \tag{enseñar}
    \tag{autónomo}
    \tag{perseverante}
    \tag{clínico}
    \tag{objetivo}
    \tag{trabajo en equipo}
    \tag{extrovertido}
    
    \section{Intereses}
    \tag{música}
    \tag{canto}
    \tag{aprender}
    \tag{enseñar}
    \tag{libertad}
    \tag{informática}
    \tag{software libre}
    \tag{minimalismo}
    \tag{hacking}
    \tag{linux}
    \tag{cine/películas}
    \tag{diseño}
    \tag{videojuegos}
    \tag{fotografía}
    \tag{filosofía}
    \tag{charlar}
    \tag{debatir}

    \section{Deseos}
    Esta pequeña sección la quiero dedicar a contar un poco más de mí.

    \medskip
    La gran mayoría de cosas que aprendí fueron de forma autodidácta; libros, internet, blogs, wikis y videos. 
    Siendo \textbf{\href{https://lukesmith.xyz/}{Luke Smith}} mi mayor mentor.

    \medskip
    También me gustaría crear una fundación para promover ideas de la libertad en general y hacer que el software libre llegue a más gente.

    \medskip
    Por otro lado más personal y no laboral... En el futuro me encantaría poder tener una banda músical y jugar mucho con el lado artístico.\\
    No descarto la idea de experimentar con líricas sobre informática y todo el mundo de la computación!

    \section{Gráfico horario}
    \wheelchart{1cm}{0.5cm}{
        6/8em/accent!20/Dormir,
        8/8em/accent!40/Escuela,
        6/8em/accent!80/Aprendiendo cosas nuevas,
        4/8em/accent!60/Ocio.
    }
}
\makebody

\end{document}
